

"Kodefejl i Sundhedsplatformen: Fem patienter har fået forkert dosis medicin"
1.
This article describes the reasons for the issue was due to codechanges
that lead to unintentional changes in the already implemented system FMK.
2. 
In order to accomodate the issue described above, it would seem obvious to initiate some procedures that ensures that code changes does not interfere with the dataintegrity.\\
Practically this would include, but not restricted to:
Thorough testing of the changes
Peer-Rewieving
Maybe running a beta version on a representive test-group
3.
The solutions presented above coincides with the somewhat vague solutions described in the article.
The article proposed that a procces is developed that ensures that no changes in "Sundhedsplatformen" can be deployed without ensuring that it does not interfere with other\\
already integrated systems.
In conclusion a combination of solutions described in 2 and 3 would seem appropriate since it accomodates the need described in the article.


"Softwareproblemer skadede mere end 100 patienter på amerikansk hospital"
1.
This article pictures does not really describe an issue.
It moreso argues that the implementation of the system, specifically the "Unknown Queue" has been done without much thought on the domain in where its used.
2.
In order so solve the issue above the system developers should definately remove the possibillity of anything going into the "unknown queue".
When any form is filled incorrectly it should be prompted to the user so they can fill it properly. A form should be submittable unless it is filled correctly.


3.
The article presents a temporary solution which consists of manually managing the "unknown queue" which seems to solve the issue for now, but is definately not a long-term solution.\\
The article presents the long term solution to be more notifications and better automatisation. More notifications coincide with the solution described above.\\
However the concept of the "unknown queue" seems to be able to cause major faults which is why we, as software engineers, think it should be removed.
In conclusion the approach we would recommend would be to work closely with an expert of the field domain in order to understand it properly.
In addition the possibillity of forms ending in the unknown queue should be removed.


4.
When developing systems in the healthcare systems some dilemmas come to light. It is of the essence to think about flaws in the software as being potentially life threatening.\\
If you develop a tool for a firm to help them track employer performance, if anything goes wrong, noone dies as a direct result of that.
However in these two cases it might just have been that someone had suffered servere physical nuisance.\\
When developing software for the healtcare system, you should always be aware that consequences of flaws might not only result in loss of money.
