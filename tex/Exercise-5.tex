Gitlet is build up in a straight forward way.
The architecture of Gitlet is not obvious. 
All the methods are listed one after another in the source code.
There are not multiple classes or any other object orinted structures. 
Considering the architecture of Git, it’s a rahter different.
Git is a system build with a bunch of modules that work together for different parts of the program.
There are a different files and folders with different parts of git inside of them, connecting the parts together to a fully build system. 
The design of Gitlet seems to be a boiled down smaller and simpler version of git.
A lot of the same features are used, but there is also a lot that got removed on purpose since the developers and the business didn’t find them as necessary. 
Regarding the different quality attributes, the most notisible are efficiency and maintainability. 
You are quickly able to run a few commands though your terminal to run through several task, that could take a lot of time otherwise. 
The maintainability in the sense that they always make sure the program is always the same, easy to use and it’s never really down or having problems since they don’t make all new big updates crashing the program or making it stop working. 
Gitlet seems to be strongly focused on efficiency and easiness of use of the program.
Gitlet too, seems to have a bigger focus on a smaller piece of software with only the most important features, so that it’s fast and easy to use.
