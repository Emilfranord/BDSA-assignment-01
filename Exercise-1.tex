\section{Exercise 1}
\paragraph{Nouns and verbs}
The following lists are of nouns and verbs, that appear in the specification.

Nouns:
\begin{enumerate}
	\item version control system
	\item changes
	\item files
	\item set of files
	\item time
	\item specific versions
	\item system
	\item source code
	\item configuration data
	\item diagrams
	\item binaries
	\item project
	\item problem
	\item issue
\end{enumerate}

Verbs:
\begin{enumerate}
	\item want
	\item records 
	\item can recall
	\item work
	\item revert
	\item compare
	\item see
\begin{enumerate}

\paragraph{Domain}
The majority of the words belong to the technical domain.
Several of the nouns only exist in the IT technical domain, such as 'source code', and 'binaries'.
Other words come form the technical world in a less direct way.

Some of the verbs that are used, relate to the concept of time.
Especially 'revert' and 'recall' belong to the time domain.

\paragraph{libgit2sharp}


% Use the noun/verb technique from "Objects First with Java: A Practical Introduction Using BlueJ" chapter 15, 
% to analyze the given description.
% Explain in which domain nouns and verbs that you identified are located.
% The implementation in libgit2sharp does neither contain a class File nor a class State. 
% Explain how that can be when libgit2sharp is an implementation of Git which is certainly a version control system as described above.
