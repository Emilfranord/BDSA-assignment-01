\section{Exercise 1}
\paragraph{Nouns and verbs}
The following lists are of nouns and verbs, that appear in the specification.

Nouns:
\begin{enumerate}
	\item version control system
	\item changes
	\item files
	\item set of files
	\item time
	\item specific versions
	\item system
	\item source code
	\item configuration data
	\item diagrams
	\item binaries
	\item state
	\item project
	\item changes
	\item problem
	\item issue
\end{enumerate}

Verbs:
\begin{enumerate}
	\item want
	\item records 
	\item can recall
	\item work
	\item revert
	\item compare
	\item see
	\item modify
	\item introduce
\begin{enumerate}

\paragraph{Domain}
The majority of the words belong to the technical domain.
Several of the nouns only exist in the IT technical domain, such as 'source code', and 'binaries'.
Other words come form the technical world in a less direct way.

Some of the verbs that are used, relate to the concept of time.
Especially 'revert' and 'recall' belong to the time domain.

All together the words belong to the domain of version control systems.

\paragraph{libgit2sharp}
While libgit2sharp does not have classes representing 'File' or 'State', it still functions as a version control system.
This is possible since the architecture of the program does not demand it.
Instead of files, this implementation tracks 'GitObject' instances, be they files or something else.
That way libgit2sharp, can still track the changes to files, and other data that needs to be tracked.
This design was probably used to abstract the implementation away from specific files and data structures.

In this design there is not a 'state', because it is an emergent property of the system.
That is, the state of the system at any given time is defined by the state of the objects, not a single object.
This was there is no need for a class to handle the state of files, commits or similar, since that is handled inside the object instances themselves.




% Use the noun/verb technique from "Objects First with Java: A Practical Introduction Using BlueJ" chapter 15, 
% to analyze the given description.
% Explain in which domain nouns and verbs that you identified are located.
% The implementation in libgit2sharp does neither contain a class File nor a class State. 
% Explain how that can be when libgit2sharp is an implementation of Git which is certainly a version control system as described above.
